% Created 2023-06-18 Sun 21:36
% Intended LaTeX compiler: pdflatex
\documentclass[11pt]{article}
\usepackage[utf8]{inputenc}
\usepackage[T1]{fontenc}
\usepackage{graphicx}
\usepackage{longtable}
\usepackage{wrapfig}
\usepackage{rotating}
\usepackage[normalem]{ulem}
\usepackage{amsmath}
\usepackage{amssymb}
\usepackage{capt-of}
\usepackage{hyperref}
\author{Rafał Grot (backend), Kamil Gunia(frontend), Karina Goszczyńska(0.1\%), Jakub Grzelec(0.1\%)}
\date{\today}
\title{Projekt PO1 ``Kalukulator Statystyk postaci na potrzeby gry RPG''}
\hypersetup{
 pdfauthor={Rafał Grot (backend), Kamil Gunia(frontend), Karina Goszczyńska(0.1\%), Jakub Grzelec(0.1\%)},
 pdftitle={Projekt PO1 ``Kalukulator Statystyk postaci na potrzeby gry RPG''},
 pdfkeywords={},
 pdfsubject={},
 pdfcreator={Emacs 30.0.50 (Org mode 9.6.1)}, 
 pdflang={English}}
\begin{document}

\maketitle
\newpage

\section{Koncepcja}
\label{sec:orgf386dec}
Celem projektu jest stworzenie aplikacji wyliczającej statystyki postaci w nie komuterowych grach RPG.
W związku z czym do wyliczenia statystyk wymagane jest dodanie statstyk.
Zakładamy że gra ma statystyki, przedmioty, stany, umiejętności, a postacie je wykorzystują.
Statystyki postaci jest zależny od np. posiadanych stanów oraz założone przedmioty.
\section{Specyfikacja, uzasadnienie wyboru konkretnych technologi.}
\label{sec:org56e82e3}
\begin{itemize}
\item Język programiowania wykorzystany do realizacji projektu \texttt{c++} został narzucony z góry.
\item Biblioteka klass wxWidgets. Ponieważ była na laboratorium. Nie wymaga nie standardowych mark preprocesora w przeciwieństwie do QT.
\item Rozdzielenie backendu oraz frontendu, jest to dobra praktyka.
\item Nie wykorzystanie bazy danych, brak umiejętności aby zrobić to prawidłowo.
\item Użycie CMake. Ponieważ używanie biblioteki wxWidgets bez multiplatformowego systemu budowy mija się z celem istnienia tej biblioteki. Dodatkowo nie ogranicza użycia różnych środowisk programistycznych.
\item System kontroli wersji git. Jest to podstawowe narzędzie programisty. Uniknięcie pytania ``Kto co robił''.
\item Doxygen do generacji dokumentacji. Był używay do projektu z podstaw programowania.
\end{itemize}
\end{document}